\documentclass[./\jobname.tex]{subfiles}
\begin{document}
%
\textbf{Abstract} This master-thesis describes and tests a differential equation solver that is based on the heuristic black-box optimisation algorithm JADE. The solver works by reformulating the residual of the differential equation into a fitness function. The underlying solution function of the differential equation is approximated by a finite sum of Radial Basis Functions. The performance of the solver is evaluated on a testbed that consists of 11 two-dimensional Poisson equations. The solving-time, the memory usage and the accuracy are compared to the open-source Finite Element Solver NGSolve as well as similar work in the current literature. To optimise the solving time and speed up the experiments, JADE is parallelised. This work introduces two new strategies: First, the number of kernels is adjusted to the differential equation during the solving process. Furthermore, a new kernel is formed and tested. These concepts are checked for significant improvements. The structure of the fitness function is investigated. A fundamental symmetry is observed that could be used to develop better suited optimisation algorithms. \\~\\
%
\noindent\rule{8.3cm}{0.4pt}

~\\~\\
\textbf{Keywords} Finite Element Methods; Weighted Residual Method; Partial Differential Equations; Numerical Methods; Differential Evolution; JADE; Downhill Simplex, Memetic Algorithm\\~\\

\noindent\rule{8.3cm}{0.4pt}

~\\~\\~\\~\\~\\~\\~\\~\\~\\~\\~\\~\\~\\~\\~\\~\\~\\~\\~\\

\noindent\rule{8.3cm}{0.4pt}

~\\~\\
Nicolai Schwartze, BSc. \\
Vorarlberg University of Applied Science \\
Master Mechatronics 2018 \\
nicolai.schwartze@students.fhv.at \\

\noindent\rule{8.3cm}{0.4pt}


\newpage
\end{document}