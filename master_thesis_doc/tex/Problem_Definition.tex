\documentclass[./\jobname.tex]{subfiles}
\begin{document}

\chapter{Problem Definition}
This chapter describes the broader idea of the \gls{wrm}, which is used to reformulate any differential equation into an optimisation problem, and its application in the field of spectral methods. The fitness function originates from these approaches. Further, the different approximation schemes and numerical solution representation are depicted. The following paragraphes also introduce the mathematical notation used for this work. 

\section{Theoretical Foundation}
\label{chap:opt_problem}

The most general description of a \gls{pde} is described in equation \ref{eq:general_pde}.
\begin{equation}
\label{eq:general_pde}
\begin{split}
\mathbf{L}u(\mathbf{x} & = f(\mathbf{x})) \\
\text{subjected to: }\mathbf{B}u(\mathbf{x} & = g(\mathbf{x})) \\
\end{split}
\end{equation}
This is similar to the formulation of the strong form in equantion \ref{eq: strong form}, but not limited to a Dirichlet boundary problem. The matrix $\mathbf{B}$ can include differentiation, effectively allowing i.e. Neumann boundary condition. 

\begin{equation}
u(\mathbf{x}) \approx u_{N}(\mathbf{x}) = \sum_{k=0}^{N} a_k \phi_k (\mathbf{x})
\end{equation}

\begin{equation}
\mathbf{R}_N(\mathbf{x}) = \mathbf{L}u_N(\mathbf{x}) - f(\mathbf{x})
\end{equation}

introduce notation: $u_{apx}(x,y) = approximated solution$, $\vec{u_{apx}} = design variables$, $u_{ext}(x,y) = exact solution$, $F(\vec{u_{apx}}) = fitness of approximated solution$

There is no easy way to approach this optimisation problem. It can not be transformed into a least squares problem, also calculating the gradient is not an option, thus justifying the usage of a heuristic algorithm.  

\section{Fitness Function}
\label{chap:fit_func}
The Fitness function is formulated as ... 

\begin{equation}
\label{eq:fit_func}
F(\vec{u}_{apx}) = \frac{\sum_{i=1}^{n_C} \xi (\mathbf{x}_i) || \mathbf{L}u(\mathbf{x}_i) - f(\mathbf{x}_i)||^2 + \phi \sum_{j=1}^{n_B} || \mathbf{B}u(\mathbf{x}_j) - g(\mathbf{x}_j)||^2}{(n_C + n_B)}  
\end{equation}

\begin{equation}
\label{eq:nc_weight}
\xi(\mathbf{x}_i) = \frac{1 + \kappa \left(1 - \frac{min_{\forall \mathbf{x}_j\in n_B}|| \mathbf{x}_i - \mathbf{x}_j ||}{max_{\forall\mathbf{x}_k \in n_C}(min_{\forall \mathbf{x}_j \in n_B} || \mathbf{x}_k - \mathbf{x}_j ||)}\right)}{1 + \kappa}
\end{equation}


\section{Candidate Representation}
\label{chap:candidate_rep}

radial basis functions with r
\begin{equation}
\label{eq: radius}
\mathbf{r} = \left|\left|\mathbf{x} - \mathbf{c} \right|\right|
\end{equation}

\subsection{Gauss Kernel}
\label{chap:gauss_kernel}

\begin{equation}
\label{eq:gauss_kernel}
gak(\mathbf{x}) = \omega e^{-\gamma \mathbf{r}^2}
\end{equation}

\begin{equation}
\label{eq:uapx_gauss_kernel}
u_{apx}(\mathbf{x}) = \sum_{i=0}^{N} \omega_i e^{-\gamma_i \mathbf{r}_i^2}
\end{equation}

\begin{equation}
\label{eq:uapx_gauss_kernel_x0}
\frac{\partial u_{apx}(\mathbf{x})}{\partial x_0} = -2 \sum_{i=0}^{N} \omega_i \gamma_i (x_0 - c_{i0}) e^{-\gamma_i \mathbf{r}_i^2}
\end{equation}

\begin{equation}
\label{eq:uapx_gauss_kernel_x0x0}
\frac{\partial^2 u_{apx}(\mathbf{x})}{\partial x_0^2} = \sum_{i=0}^{N} \omega_i \gamma_i \left[ 4 \gamma_i (x_0 - c_{i0})^2 - 2 \right] e^{-\gamma_i \mathbf{r}_i^2}
\end{equation}

\begin{equation}
\label{eq:uapx_gauss_kernel_x0x1}
\frac{\partial^2 u_{apx}(\mathbf{x})}{\partial x_0 x_1} = 4 \sum_{i=0}^{N} \omega_i \gamma_i^2 (x_0 - c_{i0}) (x_1 - c_{i1}) e^{-\gamma_i \mathbf{r}_i^2} 
\end{equation}

\subsection{GSin Kernel}
\label{chap:gsin_kernel}

\begin{equation}
\label{eq:gsin_kernel}
gsk(\mathbf{x}) = \omega e^{-\gamma \mathbf{r}^2} sin(f \mathbf{r}^2 - \varphi)
\end{equation}

\begin{equation}
\label{eq:uapx_gsin_kernel}
u_{apx}(\mathbf{x}) = \sum_{i=0}^{N} \omega_i e^{\gamma_i \mathbf{r}_i^2} sin(f_i \mathbf{r}_i^2 - \varphi_i)
\end{equation}

\begin{equation}
\label{eq:uapx_gsin_kernel_x0}
\frac{\partial u_{apx}(\mathbf{x})}{\partial x_0} = \sum_{i=0}^{N} 2 \omega_i (x_0 - c_{i0}) e^{-\gamma_i \mathbf{r}_i^2} (\gamma_i sin(\varphi_i - f_i \mathbf{r}_i^2) + f_i cos(\varphi_i - f_i \mathbf{r}_i^2))
\end{equation}

\begin{equation}
\label{eq:uapx_gsin_kernel_x0_x0}
\begin{split}
& \frac{\partial^2 u_{apx}(\mathbf{x})}{\partial x_0^2} = \sum_{i=0}^{N} 2 \omega_i e^{-\gamma_i \mathbf{r}_i^2} \\ & [ (2 c_{i0}^2 (f_i^2 - \gamma_i^2) + 4 c_{i0} x (\gamma_i^2 - f_i^2) + 2 f_i^2 x^2 - 2 \gamma_i^2 x^2 + \gamma_i) sin(\varphi_i - f_i \mathbf{r}_i^2) + \\ & f_i (-4 c_{i0}^2 \gamma_i + 8 c_{i0} \gamma_i x - 4 \gamma_i x^2 + 1) cos(\varphi_i - f_i \mathbf{r}_i^2) ]
\end{split}
\end{equation}

\begin{equation}
\label{eq:uapx_gsin_kernel_x0_x1}
\begin{split}
& \frac{\partial^2 u_{apx}(\mathbf{x})}{\partial x_0 x_1} = \sum_{i=0}^{N}  4 \omega_i (c_{i0} - x) (c_{i1} - y) e^{-\gamma_i \mathbf{r}_i^2} \\ & \left[(f_i^2 - \gamma_i^2) sin(\varphi_i - f_i \mathbf{r}_i^2) - 2 f_i \gamma_i cos(\varphi_i - f_i \mathbf{r}_i^2)\right]
\end{split}
\end{equation}

\end{document}