\documentclass[./\jobname.tex]{subfiles}
\begin{document}
\chapter {Experiment 0: Serial Memetic JADE}
\label{chap:experimet_0}

This chapter describes the results obtained by the most basic adaption of JADE for solving \gls{pde}s. The algorithm used here, builds on the concepts described in \cite{chaquet_using_2019}. Aside from substituting some parameters, the only main difference is the usage of JADE instead of a \gls{cma_es}. 

with $10^6$ function evaluation: \\
hyphotosis test in the difference of L2 norm reached \\
statistically relevant vs. practically relevant -> show 3D plot of what the difference in L2 norm looks like (= effect size) \\
this servers as an experimental prove, that the objective function is multimodal and DS runs into premature convergence \\

show convergence on PDE0A \\
Compare to Chaquet Results PDE2 and PDE3\\

with $10^4$ function evaluation: \\
box plot of time and memory usage -> compare to FEM \\
L2 norm table reached \\
how does time/memory scale with number of FE? -> O(n)\\
memory not needed: only for evaluation -> could be smaller \\
what can be done about time? -> transition to next chapter: parallel
what are simpler and more complex testbed pdes? \\

\section{Hypotheses}

\begin{algorithm}[H]
	\SetAlgoNoLine
	\DontPrintSemicolon
	\SetKwFunction{FmJADE}{memeticJADE}
	\SetKwProg{Fn}{Function}{:}{}
	\Fn{\FmJADE{$\mathbf{X}$, $funct$, $minErr$, $maxFE$}}{
		$dim$, $popsize$ $\gets size(\mathbf{X})$\;
		$p \gets 0.3$\;
		$c \gets 0.5$\;
		$pop$, $FE$, $F$, $CR$ $\gets JADE($$\mathbf{X}$, $p$, $c$, $funct$, $minErr$, $maxFE - 2 dim$ $)$\;
		$bestIndex = argmin(FE)$\;
		$bestSol = pop[bestIndex]$\;
		$pop$, $FE$ $ = downhill\text{ }simplex($$funct$, $bestSol$, $minErr$, $2 dim)$
		\Return $pop$, $FE$, $F$, $CR$
	}
	\unterschrift{memeticJADE Pseudocode}{}{}
	\label{algo: memeticJADE}
\end{algorithm}

\section{Experiment Setup}

\section{Result}
Table \ref{tab:results_literature_comparison} shows a compariosn on the common testbed functions \gls{pde} 2 and \gls{pde} 3 with numerical results obtained in other research papers. It is important to notice, that the parameters, used to obtain the results, might not coinside. \cite{tsoulos_solving_2006} also use these two \gls{pde}, but their paper did not provide any usable error metric. 
\begin{table}[h]
	\centering
	\noindent\adjustbox{max width=\linewidth}{
		\begin{tabular}{|c|c|c|c|}
			
			\hline
			\rowcolor[HTML]{\farbeTabA}
			
			Paper & parameter & RMSE \gls{pde} 2 & RMSE \gls{pde} 3 \\ \hline
			\cite{chaquet_using_2019} & \multilinecell{4 kernel \\ max \#FE=$10^6$ \\ 50 replications } & $(1.75 \pm 1.14) 10^{-4}$ & $(1.09 \pm 0.846) 10^{-5}$  \\ \hline
			\cite{chaquet_solving_2012} & \multilinecell{10 harmonics \\ max \#FE = $G \cdot \lambda$ = $1.2 \cdot 10^6$ \\ 10 replications} & $(6.37 \pm 0.733) 10^{-3}$ & $(5.90 \pm 0.799)10^{-3}$ \\ \hline
			\cite{sobester_genetic_2008}& \multilinecell{50 max tree lenght \\ 12 generations \\ 20 replications} & $(6.9 \pm 8.3)10^{-4}$ & - \\ \hline
			\cite{panagant_solving_2014}& \multilinecell{unknowns: N/A \\ \#FE=$5\cdot 10^5$ \\ replications: N/A} & $7.256 10^{-4}$ & $9.489 10^{-6}$ \\ \hline
			serial memetic JADE & \multilinecell{5 kernel \\ max \#FE = $10^6$ \\ 20 replications} & & \\ \hline
			
			
		\end{tabular}
	}
	\unterschrift{This table compares the results obtained with the serial memetic JADE to the numerical results obtained by similar work in literature. The same metric must be used, thus the RMSE as defined in equation \ref{eq:rmse_chaquet} is calculated.}{}{}
	\label{tab:results_literature_comparison}
\end{table}


\begin{table}[h]
	\centering
	\noindent\adjustbox{max width=\linewidth}{
		\begin{tabular}{|c|c|c|c|c|c|}
			
			\hline
			\rowcolor[HTML]{\farbeTabA}
			
			\#FE & \multicolumn{2}{|c|}{$10^4$} & \multicolumn{2}{|c|}{$10^6$} & \\ \hline
			stat & mean & median & mean & median & Wilcoxon Test \\ \hline \hline
			\gls{pde} 0A & 1.9415 $\pm$ 0.3321 & 1.8844 & 0.6596 $\pm$ 0.5510 & 0.9285 & sig. better \\ \hline
			\gls{pde} 0B & 0.7137 $\pm$ 0.1979 & 0.6354 & 0.2027 $\pm$ 0.1302 & 0.1516 & sig. better \\ \hline
			\gls{pde}  1 & 0.1874 $\pm$ 0.0408 & 0.1938 & 0.0149 $\pm$ 0.0049 & 0.0151 & sig. better \\ \hline
			\gls{pde}  2 & 0.0890 $\pm$ 0.0334 & 0.0760 & 0.0257 $\pm$ 0.0140 & 0.0224 & sig. better \\ \hline
			\gls{pde}  3 & 0.2409 $\pm$ 0.1051 & 0.2309 & 0.0328 $\pm$ 0.0169 & 0.0285 & sig. better \\ \hline
			\gls{pde}  4 & 0.1102 $\pm$ 0.0367 & 0.0985 & 0.0378 $\pm$ 0.0083 & 0.0352 & sig. better \\ \hline
			\gls{pde}  5 & 0.6645 $\pm$ 0.1930 & 0.6263 & 1.1968 $\pm$ 0.0286 & 1.2056 & sig. worse \\ \hline
			\gls{pde}  6 & 1.9660 $\pm$ 1.3845 & 1.6540 & 0.4135 $\pm$ 1.2133 & 0.0018 & sig. better \\ \hline
			\gls{pde}  7 & 0.0457 $\pm$ 0.0137 & 0.0452 & 0.0221 $\pm$ 0.0019 & 0.0223 & sig. better \\ \hline
			\gls{pde}  8 & 0.2186 $\pm$ 0.0045 & 0.2191 & 0.2170 $\pm$ 0.0019 & 0.2175 & unsig. better \\ \hline
			\gls{pde}  9 & 0.0525 $\pm$ 0.0147 & 0.0516 & 0.0451 $\pm$ 0.0119 & 0.0459 & unsig. better \\ \hline
			
		\end{tabular}
	}
	\unterschrift{}{}{}
	\label{tab:compare_10^6_10^4}
\end{table}


\section{Interpretation}



\end{document}