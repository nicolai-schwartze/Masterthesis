%
% 1: documentation, e.g.: a documentation for a project done in a course
% 2: thesis, e.g.: a master thesis
% 3: summary
% 5: presentation
% 9: paper
%
\ifx\FHVmode\undefined
	\def\FHVmode{2}
\fi

%
% 1: Presentation only
% 2: Presentation only notes
% 3: Presentation all (left page is the presentation, right the notes)
%
\ifx\PresentationMode\undefined
	\def\PresentationMode{3}
\fi
%
\def\paperStyle{fhv}
%
\newcommand{\version}{v0.0}
%
\pdfminorversion=7 % use newer pdf version
\pdfobjcompresslevel=2
\pdfsuppresswarningpagegroup=1
%
\input{./sty/overallDefines.sty}
%
\documentclass[\documentclassOptions]{\komaScriptClass}
%
\usepackage{./sty/fhv}
%
\makeglossaries
\renewcommand{\glsnamefont}[1]{\textbf{#1}}
%
% General Settings for Title...
%

% colour of hyperlinks instead of boxes
\hypersetup{
	colorlinks   = true,  %Colours links instead of ugly boxes
	urlcolor     = blue,  %Colour for external hyperlinks
	linkcolor    = black, %Colour of internal links
	citecolor    = black  %Colour of citations
}

\setLanguage{\newLanguage}
\setTitle{Title}
\setThesistype{Master Thesis}
\setAuthor{\authorSurname\authorName}
\setAuthorId{MATRIKELNUMMER}
\setStudyprogram{Master's in Mechatronics}
\setSupervisor{\supervisorSurname\supervisorName}
\setSupervisorCompany{Title B SupervisorCompanyName, Title A}
\setSubtitle{Subtitle}
\setSubject{Subject}
\setDegree{Master of Science in Engineering, MSc}
\setCompany{Company Name GmbH}
%
% PDF Settings
%
\pdfSettings
%
\allowdisplaybreaks
%
% hack to get the caption wider for a table
%
\renewcommand{\TPTminimum}{\linewidth}
%
\begin{document}
	%
	% Select the language defined in \newLanguage
	%
	\ifx\newLanguage\ngerman
		\selectlanguage{ngerman}
	\else
		\selectlanguage{english}
	\fi %
	%
	\if\FHVmode\paper
		\SetAlgorithmName{Algorithmus}{Alg.}
	\makeatletter
		\crefname{equation}{Gl.}{Gln.}
	\makeatother
	\fi
	% Import the acronyms
	%
	\newacronym{de}{DE}{differential evolution}
\newacronym{ge}{GE}{grammatical evolution}
\newacronym{ga}{GA}{genetic algorithm}
\newacronym{pso}{PSO}{particle swarm optimisation}
\newacronym{gp}{GP}{genetic programming}
\newacronym{ode}{ODE}{ordinary differential equation}
\newacronym{pde}{PDE}{partial differential equation}
\newacronym{fem}{FEM}{finite element method}
\newacronym{bvp}{BVP}{boundary value problem}
\newacronym{wca}{WCA}{Water Cycle Algorithm}
\newacronym{hs}{HS}{Harmony Search}
\newacronym{rbf}{RBF}{radial basis function}
\newacronym{ab}{AB}{Adams–Bashforth}
\newacronym{fe}{FE}{Forward-Euler}
\newacronym{rk}{RK}{Runge–Kutta}
\newacronym{wrm}{WRM}{weighted residual method}
\newacronym{ci}{CI}{Computational Intelligence}
\newacronym{cma_es}{CMA-ES}{Covariance Matrix Adaption Evolution Strategy}
\newacronym{es}{ES}{Evolution Strategy}
\newacronym{ds}{DS}{Downhill-Simplex}
\newacronym{nm}{NM}{Nelder-Mead}
\newacronym{rbfnn}{RBF-NN}{radial basis function neuronal network}
\newacronym{uml}{UML}{Unified Modeling Language}
\newacronym{rss}{RSS}{Resident Set Size}
\newacronym{vms}{VMS}{Virtual Memory Size}
\newacronym{rmse}{RMSE}{Root-Mean-Square Error}
\newacronym{rhs}{RHS}{right hand side}
\newacronym{dof}{DOF}{Degree of Freedom}
\newacronym{gui}{GUI}{Graphical User Interface}
\newacronym[sort=hash]{nfe}{\#FE}{Number of Function Evaluation}
\newacronym{json}{JSON}{JavaScript Object Notation}
\newacronym{gak}{GaK}{Gauss Kernel}
\newacronym{gsk}{GSK}{Gauss Sine Kernel}
\newacronym{sp}{sp}{speed-up}
\newacronym{dt}{dT}{Delay Time} % provide the defined acronyms to be used
	%
	% for Backlinks to work properly
	%
	\begin{envModeNot}[\presentations]
		%
		% for Backlinks to work properly
		%
		\let\hypercontentsline=\contentsline
		\renewcommand{\contentsline}[4]{\hypertarget{toc.#4}{}\hypercontentsline{#1}{#2}{#3}{#4}}%
		%
		\sisetup{output-decimal-marker = {,}}
		\pagenumbering{gobble} % used to prevent the page numbering
		%
		\begin{envDebug}
			\layout
			\textrm{Serif: \rmdefault}\par
			\textsf{Sans-Serif: \sfdefault}\par
			\texttt{Teletype: \ttdefault}
		\end{envDebug}
		%
		\begin{envModeNot}[\paper]
		%
		% evtl. Sperrvermerkseite
		% nur in begründeten Ausnahmefällen verwenden
		% Aufgrund gesetzlicher Bestimmungen ist eine Sperre maximal für fünf Jahre möglich
		%
		%\sperrvermerk{5}
		%
		\end{envModeNot}
		%
		\maketitle % creates the title page
		\hypersetup{pageanchor=true}
		%
		\begin{envModeNot}[\paper]
		%
		\pagenumbering{Roman} 
		%
		% Abstracts
		
		\subfile{./tex/Kurzreferat.tex}
		\subfile{./tex/Abstrait.tex}
		\subfile{./tex/Abstract.tex}
		
		\newpage
		%
		\fhvlists
		\end{envModeNot}
		%
		% INSERT your .tex files
		%
		\subfile{./tex/Introduction.tex}
		\subfile{./tex/State_of_the_Art.tex}
		\subfile{./tex/Problem_Definition.tex}
		\subfile{./tex/Experimental_Design.tex}
		\subfile{./tex/Experiment0.tex}
		\subfile{./tex/Experiment1.tex}
		\subfile{./tex/Experiment2.tex}
		\subfile{./tex/Experiment3.tex}
		\subfile{./tex/Experiment4.tex}
		\subfile{./tex/Limitations.tex}
		\subfile{./tex/Conclusion.tex}
		\subfile{./tex/Summary.tex}
		%
		% END INSERT
		%
		\glossaryAndBibliography
		%
		% uncomment this if you like a short CV
		% \subfile{./tex/Lebenslauf.tex}
		%
		\begin{envModeNot}[\paper]
			\newpage
			\appendix
			\addAppendix{
			%
			% INSERT your .tex files
			%
			\subfile{./tex/Appendix.tex}
			%
			% END
			%
			}
			% Statuory Declaration
			\statuoryDeclaration
			%
		\end{envModeNot}
	\end{envModeNot}
	%
\end{document}
