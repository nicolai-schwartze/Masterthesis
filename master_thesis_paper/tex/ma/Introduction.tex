\documentclass[./\jobname.tex]{subfiles}
\begin{document}
\chapter{Introduction}

Differential equations are a very powerful mathematical tool to describe the universe. From fluid dynamic and heat flow in mechanical engineering or electrodynamics and circuit-design in electrical engineering to the fluctuation of stock market prices and even the movement of astronomical objects - most dynamic processes can be modelled by them. But only a tiny fraction of them are currently analytically solvable. The rise of the computer paved the way for approximating the solution of a differential equation numerically. There are plenty of different solver methods ranging from simple single step solvers like the \gls{fe} method over more complex multistep solvers like the \gls{ab} method to whole Finite Element Method (FEM) solver packages. Most of these solvers are specialised for one kind of differential equation, leveraging some special properties of the problem to be most efficient in time and error. But there are very few generalised methods that can solve many different types of equations. This lack of a universal solver is the main motivation of the present master thesis.

There is already a small, yet steadily growing research community interested in tying together the concepts of computational intelligence and numerical solver for differential equations. In chapter \ref{chap:literature_overview} a brief overview of related work done within the last 20 years is given. The main idea of all listed papers is to reformulate the original problem into an optimisation problem, which in turn is then solved using an evolutionary algorithm. The main focus of this thesis lies on two-dimensional Partial Differential Equations (PDE). The results are then compared to the analytical solution as well as a state of the art FEM solver.  

This master thesis tries to answer the following questions: 
Is it possible to improve the results obtained in other research papers by using a modern variant of the \gls{de} optimisation algorithm? How well does this method stack up against classical FEM packages for solving PDEs considering time and memory usage as well as numerical error? Is there a meaningful way to reduce the time consumption by making use of a parallel computation architecture? 

\end{document}