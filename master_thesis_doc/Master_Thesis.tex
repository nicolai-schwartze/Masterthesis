%
% 1: documentation, e.g.: a documentation for a project done in a course
% 2: thesis, e.g.: a master thesis
% 3: summary
% 5: presentation
% 9: paper
%
\ifx\FHVmode\undefined
	\def\FHVmode{2}
\fi

%
% 1: Presentation only
% 2: Presentation only notes
% 3: Presentation all (left page is the presentation, right the notes)
%
\ifx\PresentationMode\undefined
	\def\PresentationMode{3}
\fi
%
\def\paperStyle{fhv}
%
\newcommand{\version}{v0.0}
%
\pdfminorversion=7 % use newer pdf version
\pdfobjcompresslevel=2
\pdfsuppresswarningpagegroup=1
%
%
% only \def and \newcommand{cmd}{def} allowed here!
% because no packages are loaded here
%
\def\pathName{Programm}
\def\ValueBindingOffset{0mm} 
\def\debug{false}
\def\documentation{1}
\def\thesis{2}
\def\summary{3}
\def\presentations{5}
\def\paper{9}
\def\PresentationOnly{1}
\def\PresentationNotes{2}
\def\PresentationAll{3}
\def\ngerman{ngerman}
% define if you want to show
\def\notesFHV{draft}   % disable: notes not showed
                                    % draft:     notes showed
%
% Uncomment this, if you want another styled title page
%
%\def\FHVtitlePage{fhv}
%
% english or ngerman
%
\ifx\newLanguage\undefined
%	\def\newLanguage{ngerman} % build twice!!!
	\def\newLanguage{english} % build twice!!!
\fi
%
% Defines to make live easier...
% ATTENTION: do not forget to put a \xspace after the command!
%
\def\pfeil{\ensuremath{\rightarrow}\xspace}
\def\authorName{Nicolai\xspace}
\def\authorSurname{Schwartze\xspace}
% uncomment \authorTitleBefore & \authorTitleAfter if not used
%\def\authorTitleBefore{\xspace}
\def\authorTitleAfter{BSc\xspace}
\def\supervisorName{Steffen\xspace}
\def\supervisorSurname{Finck\xspace}
% uncomment \supervisorTitleBefore & \supervisorTitleAfter if not used
\def\supervisorTitleBefore{Dr.-Ing.\xspace}
%\def\supervisorTitleAfter{\xspace}
%
% uncomment this if you like a title, e.g. for the mechatronik plattform
%
%\def\paperName{KONFERENZ DER MECHATRONIK-PLATTFORM: Smart Factory\\
%							FH Technikum Wien,  24. November 2016\xspace}
%
% Define if the author is male (m) or female (w)
%
\def\wOrM{m}
%
% pdf settings
%
\makeatletter
	\immediate\write18{git log -1 --pretty=format:"\@backslashchar gdef\@backslashchar GITHash{\@percentchar h}\@percentchar n\@backslashchar gdef\@backslashchar GITDate{\@percentchar ad}" --date=short > build/git-info.txt}
\makeatother
\input{build/git-info.txt}
%
\def\pdfSettings{%
	%
	\def\tmpPdfSubject{\ifdefined\GITHash%
		git-short:~\GITHash%
	\fi\ifdefined\GITDate%
		\xmpcomma~git-date:~\GITDate%
	\fi}
	\hypersetup{pdftitle={\getTitle},%
		pdfauthor={\authorSurname\xmpcomma~\authorName},%
		pdfsubject={\tmpPdfSubject},%
		pdfkeywords={Fachhochschule Vorarlberg : \getThesistype~: \studyType~: \supervisorSurname, \supervisorName},%
		pdflang={de},%
		unicode=true,}%
}
%
% standard definiton for biblatex
%
\def\biblatexOptions{
	backend=biber,%
	style = authoryear,%
	citestyle = authoryear,
	dashed=false,
	backref=true,
}
%
% standard definiton for the document class
%
\def\komaScriptClass{scrreprt}%
\def\documentclassOptions{%
	a4paper,%
	oneside,%
	fontsize=11pt,%
	DIV=calc,%
	headsepline,%
	%%BCOR=10mm,
%	parskip=half,%
	headings=big,%
	draft=\debug,%
	numbers=noenddot,%
%	toc=sectionentrywithdots,%
%	chapterentrydots=true,%
	listof=totoc,%
	%bibliography=totoc,%
	%headinclude=false, %koma script seite 79 tabelle 3.7
}
%
\if\paper\FHVmode
	\def\komaScriptClass{scrartcl}%
	\def\documentclassOptions{%
		a4paper,%
		oneside,%
		fontsize=9pt,%
		DIV=calc,%
%		draft=\debug,%
		numbers=enddot,%
		twocolumn,
		bibliography=totocnumbered,%
	}%
	\def\biblatexOptions{
		backend=biber,%
		style = authoryear,%
		citestyle = authoryear,%
		dashed=false,%
	}
\fi
%
\if\summary\FHVmode
%\def\komaScriptClass{scrartcl}%
\def\komaScriptClass{scrreprt}%
\def\documentclassOptions{%
	a4paper,%
	oneside,%
	fontsize=11pt,%
	DIV=calc,%
%	parskip=half,%
	headings=big,%
	draft=\debug,%
	numbers=noenddot,%
	%	toc=sectionentrywithdots,%
	%	chapterentrydots=true,%
	listof=totoc,%
	numbers=noenddot,%
}
\fi%
%
\if\presentations\FHVmode
\def\komaScriptClass{beamer}%
\def\documentclassOptions{%
	18pt,%
	xcolor=dvipsnames,%
	hyperref={breaklinks=true},%
	xcolor=table,%
}%
\fi
%
% Color for tables
% 
\def\farbeTabA{C0C0C0}
\def\farbeTabB{EFEFEF}
%
% Commands to make live easier...
%
\newcommand{\mtnote}[1]{\textsuperscript{\TPTtagStyle{#1}}}
%
\def\SymbReg{\textsuperscript{\textregistered}}
\newcommand{\MATLAB}{\textsc{Matlab\small\SymbReg}\xspace}
%
% do the caption with a source
%
\newcommand{\unterschrift}[3]%
{%
	\ifthenelse{\equal{\getLanguage}{english}}% english or german
	{\def\quelle{Source}}%
	{\def\quelle{Quelle}}%
	\def\source{\ifthenelse{\equal{#2}{}}{}{\\\quelle: #2}}%
	% if no source is given, don't use this
%	\citetrackerfalse\pagetrackerfalse\backtrackerfalse%
	\ifthenelse{\equal{#3}{no}}% add to the table of contents? Standard is on
	{\caption[]{#1\source}}%
	{\caption[#1]{#1\source}}%
%	\citetrackertrue\pagetrackertrue\backtrackertrue%
}%
%
\newcommand{\FHVcheckbox}[1]%
{%
	\def\checked{1}
	\def\unchecked{0}
	\if#1\checked
		\makebox[0pt][l]{$\square$}\raisebox{.15ex}{\hspace{0.1em}\(\checkmark\)}
	\fi
	\if#1\unchecked
		\makebox[0pt][l]{$\square$}\raisebox{.15ex}{\hspace{0.1em}}\hspace{1em}
	\fi
}
%

%
\documentclass[\documentclassOptions]{\komaScriptClass}
%
\usepackage{./sty/fhv}
%
\makeglossaries
\renewcommand{\glsnamefont}[1]{\textbf{#1}}
%
% General Settings for Title...
%

% colour of hyperlinks instead of boxes
\hypersetup{
	colorlinks   = true,  %Colours links instead of ugly boxes
	urlcolor     = blue,  %Colour for external hyperlinks
	linkcolor    = black, %Colour of internal links
	citecolor    = black  %Colour of citations
}

\setLanguage{\newLanguage}
\setTitle{Title}
\setThesistype{Master Thesis}
\setAuthor{\authorSurname\authorName}
\setAuthorId{MATRIKELNUMMER}
\setStudyprogram{Master's in Mechatronics}
\setSupervisor{\supervisorSurname\supervisorName}
\setSupervisorCompany{Title B SupervisorCompanyName, Title A}
\setSubtitle{Subtitle}
\setSubject{Subject}
\setDegree{Master of Science in Engineering, MSc}
\setCompany{Company Name GmbH}
%
% PDF Settings
%
\pdfSettings
%
\allowdisplaybreaks
%
% hack to get the caption wider for a table
%
\renewcommand{\TPTminimum}{\linewidth}
%
\begin{document}
	%
	% Select the language defined in \newLanguage
	%
	\ifx\newLanguage\ngerman
		\selectlanguage{ngerman}
	\else
		\selectlanguage{english}
	\fi %
	%
	\if\FHVmode\paper
		\SetAlgorithmName{Algorithmus}{Alg.}
	\makeatletter
		\crefname{equation}{Gl.}{Gln.}
	\makeatother
	\fi
	% Import the acronyms
	%
	\newacronym{de}{DE}{Differential Evolution}
\newacronym{ge}{GE}{Grammatical Evolution}
\newacronym{ga}{GA}{Genetic Algorithm}
\newacronym{pso}{PSO}{Particle Swarm Optimisation}
\newacronym{gp}{GP}{Genetic Programming}
\newacronym{ode}{ODE}{Ordinary Differential Equation}
\newacronym{pde}{PDE}{Partial Eifferential Equation}
\newacronym{fem}{FEM}{Finite Element Method}
\newacronym{bvp}{BVP}{Boundary Value Problem}
\newacronym{wca}{WCA}{Water Cycle Algorithm}
\newacronym{hs}{HS}{Harmony Search}
\newacronym{rbf}{RBF}{Radial Basis Function}
\newacronym{ab}{AB}{Adams–Bashforth}
\newacronym{fe}{FE}{Forward-Euler}
\newacronym{rk}{RK}{Runge–Kutta}
\newacronym{wrm}{WRM}{Weighted Residual Method}
\newacronym{ci}{CI}{Computational Intelligence}
\newacronym{cma_es}{CMA-ES}{Covariance Matrix Adaption Evolution Strategy}
\newacronym{es}{ES}{Evolution Strategy}
\newacronym{ds}{DS}{Downhill-Simplex}
\newacronym{nm}{NM}{Nelder-Mead}
\newacronym{rbfnn}{RBF-NN}{Radial Basis Function Neuronal Network}
\newacronym{uml}{UML}{Unified Modeling Language}
\newacronym{rss}{RSS}{Resident Set Size}
\newacronym{vms}{VMS}{Virtual Memory Size}
\newacronym{rmse}{RMSE}{Root-Mean-Square Error}
\newacronym{rhs}{RHS}{Right Hand Side}
\newacronym{dof}{DOF}{Degree of Freedom}
\newacronym{gui}{GUI}{Graphical User Interface}
\newacronym[sort=hash]{nfe}{\#FE}{Number of Function Evaluation}
\newacronym{json}{JSON}{JavaScript Object Notation}
\newacronym{gak}{GaK}{Gauss Kernel}
\newacronym{gsk}{GSK}{Gauss Sine Kernel}
\newacronym{sp}{sp}{Speed-Up}
\newacronym{dt}{dT}{Delay Time}
\newacronym{erd}{ERD}{Empirical Runtime Distribution}
\newacronym{bbob}{BBOB}{Black Box Optimisation Benchmarking}
\newacronym{coco}{COCO}{Comparing Continuous Optimisers} % provide the defined acronyms to be used
	%
	% for Backlinks to work properly
	%
	\begin{envModeNot}[\presentations]
		%
		% for Backlinks to work properly
		%
		\let\hypercontentsline=\contentsline
		\renewcommand{\contentsline}[4]{\hypertarget{toc.#4}{}\hypercontentsline{#1}{#2}{#3}{#4}}%
		%
		\sisetup{output-decimal-marker = {,}}
		\pagenumbering{gobble} % used to prevent the page numbering
		%
		\begin{envDebug}
			\layout
			\textrm{Serif: \rmdefault}\par
			\textsf{Sans-Serif: \sfdefault}\par
			\texttt{Teletype: \ttdefault}
		\end{envDebug}
		%
		\begin{envModeNot}[\paper]
		%
		% evtl. Sperrvermerkseite
		% nur in begründeten Ausnahmefällen verwenden
		% Aufgrund gesetzlicher Bestimmungen ist eine Sperre maximal für fünf Jahre möglich
		%
		%\sperrvermerk{5}
		%
		\end{envModeNot}
		%
		\maketitle % creates the title page
		\hypersetup{pageanchor=true}
		%
		\begin{envModeNot}[\paper]
		%
		\pagenumbering{Roman} 
		%
		% Abstracts
		
		\subfile{./tex/Kurzreferat.tex}
		\subfile{./tex/Abstrait.tex}
		\subfile{./tex/Abstract.tex}
		
		\newpage
		%
		\fhvlists
		\end{envModeNot}
		%
		% INSERT your .tex files
		%
		\subfile{./tex/Introduction.tex}
		\subfile{./tex/State_of_the_Art.tex}
		\subfile{./tex/Problem_Definition.tex}
		\subfile{./tex/Experimental_Design.tex}
		\subfile{./tex/Experiment0.tex}
		\subfile{./tex/Experiment1.tex}
		\subfile{./tex/Experiment2.tex}
		\subfile{./tex/Experiment3.tex}
		\subfile{./tex/Experiment4.tex}
		\subfile{./tex/Limitations.tex}
		\subfile{./tex/Conclusion.tex}
		\subfile{./tex/Summary.tex}
		%
		% END INSERT
		%
		\glossaryAndBibliography
		%
		% uncomment this if you like a short CV
		% \subfile{./tex/Lebenslauf.tex}
		%
		\begin{envModeNot}[\paper]
			\newpage
			\appendix
			\addAppendix{
			%
			% INSERT your .tex files
			%
			\subfile{./tex/Appendix.tex}
			%
			% END
			%
			}
			% Statuory Declaration
			\statuoryDeclaration
			%
		\end{envModeNot}
	\end{envModeNot}
	%
\end{document}
