\documentclass[./\jobname.tex]{subfiles}
\begin{document}
\chapter{Further Work}

The current implementation is not complete. Many improvements could be performed to reduce the time consumption and potentially increase the accuracy. They can be grouped into three major topics.  

\underline{\textbf{Technology:}} \\
At the moment, the solver is implemented in Python. Since Python is an interpreted scripting language it is easy to use but it is also slow. Pre-compiling the optimisation algorithm and the fitness function could decrease the time consumption. This would also mean that the \gls{nfe} could be increased, which could even enhance the accuracy.

\underline{\textbf{Theory:}}\\
The described symmetry should be further investigated. Designing a variational operator to that circular feature could significantly improve the searching process. Also the search space could be restricted so that not the whole $\mathbb{R}^N$ is taken into account. It might even be a starting point for further theoretical advances.

Generally, more mathematical rigour should be applied to this solver strategy. Users are rightfully reluctant to apply a solver to their problem if no convergence properties are proven. 

\underline{\textbf{Algorithm:}}\\
It seems that JADE can not successfully substitute the \gls{cma_es}. To explore the limitations of a \gls{cma_es}, it should be applied on the current testbed - especially the hard problems like \gls{pde} 5. 

In a next step the \gls{cma_es} could be tested in combination with the newly defined concept of the kernel adaption scheme. Faster convergence could effectively overcome the problems exhibited with JADE. Further, the \gls{cma_es} and the new \gls{gsk} could be tied together. It is possible that the \gls{cma_es} is better at handling the higher dimensions of the \gls{gsk}. 

The \gls{cma_es} (year 2003) is the backbone of many \gls{es} based optimisation techniques. However, the bare \gls{cma_es} is already a few years old and many new iterations are proposed. It might be possible to increase the performance by using one of these algorithms. 

In \gls{fem}, the adaptive mesh refinement is used to increase the accuracy at certain areas of the \gls{pde}. Similarly, the collocation points of the \gls{ci} solver could be adapted. Where the fitness value is greater, new collocation points could be included. This would set the emphasis on these areas which would then be subjected to refinement. 





\end{document}