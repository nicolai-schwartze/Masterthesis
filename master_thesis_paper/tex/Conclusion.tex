\documentclass[./\jobname.tex]{subfiles}
\begin{document}
\section{Conclusion}
Using JADE could not replicate the objectively good performance observed with other related work. The reason for this is not completely apparent. The poor convergence might be due to some bad parameter choices. Otherwise, JADE might not be as well suited for that sort of fitness function as other optimisation algorithms. To speed up the optimisation process, the parallel JADE is implemented. Thus, the already present parallel hardware structure can be used. The parallel JADE works as expected, however it does not reduce the computational effort. Since nowadays, single-core machines do not exist any more, a parallel structure should be the standard for all newly created software. The purpose of the adaptive kernels scheme is to reduce the search dimension and increase the number of kernels only when necessary. It works very well on the testbed \gls{pde} 0A. This \gls{pde} fulfils the assumptions that have been made for this algorithm. Mixed results have been observed on all other testbed problems. The experiments suggest that the problems could be overcome with an algorithm that converges faster. The present fitness function might not be the perfect formulation of the problem. Ideally, the quality of the approximation should be tightly linked to the fitness value. However, experiments have shown that this might not always be the case. Specifically, on \gls{pde} 5 more \gls{nfe} result in a significantly worse approximation quality. Thus, a new kernel called \gls{gsk} is created. The goal for this kernel is to alter the fitness function and thus enhance the correlation between fitness function and quality. The idea works as intended and the results on that \gls{pde} do get significantly better. Although the \gls{gsk} exhibits all the features of the Gauss kernel, the quality of the results for all other \gls{pde}s are mixed. This might be due to the larger search space that comes with the new kernel. More experiments investigating the connection of the adaptive scheme with the \gls{gsk} could provide more insight into that subject. Obviously, this \gls{ci} approach to solve \gls{pde}s is not as effective as the \gls{fem} method. \gls{fem} is specifically designed to work on elliptic \gls{pde}s and convergence is proven. However, the \gls{ci} method is a universal approach that should be able to solve all sorts of differential equations. Contrary, the ``universal solver'' attribute might also be the greatest disadvantage of this strategy. Currently, there is no guarantee for convergence. The simple implementation and the relatively straight forward approach could misguide the user into applying this strategy, although there might exist better solvers for the problem at hand. The described symmetry should be further investigated. This could lead to the design of an algorithm that is explicitly tailored to this optimisation problem. It might even be a starting point for further theoretical advances.

\end{document}